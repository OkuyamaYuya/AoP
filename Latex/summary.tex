\documentclass[11pt,a4paper]{jarticle}
\usepackage[dvipdfmx]{graphicx}
\usepackage{amsmath}
\usepackage{here}
\usepackage{indentfirst}
%
\topmargin=-5mm
\oddsidemargin=-5mm
\evensidemargin=-5mm
\textheight=235mm
\textwidth=165mm
%
\title{中間発表}
\author{奥山 裕也\\
        学際科学科 総合情報学コース\\
        08-152020\\
       }
\date{}
%
\begin{document}
\maketitle
\section*{題目}
最適化問題を解くプログラムを順序と生成関数より自動導出する研究 : 動的計画法と貪欲法について
\section*{指導教員}
森畑明昌 講師
\section*{要旨}
本研究は、組合わせ最適化問題の解を求める効率のよいプログラムを書くことを補助するシステムの開発を目標としている。このシステムを利用するとき、プログラマはどのアルゴリズムを用いるか悩む必要がなく、問題の候補解を列挙する関数(生成関数)と解の大小を比べる順序を記述して最適化問題を定式化するだけでよい。そしてコンパイル時、問題が貪欲法で解けるか動的計画法で解けるかどちらも適用できないかが判定されプログラムが導出される。判定は、順序の単調性の成立をSMTソルバーなどを用いて調べることで行われる。またライブラリの実装は、R. S. BirdとO. de Moor \cite{aop}によって示された動的計画法定理と貪欲法定理を元にして行う。
\begin{thebibliography}{9}
\bibitem{aop} R. S. Bird, O. de Moor, Algebra of Programming, Prentice Hall, 1997.
\end{thebibliography}
\end{document}